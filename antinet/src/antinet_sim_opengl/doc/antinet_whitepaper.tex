\documentclass[a4paper,11pt]{article}
\usepackage[T1]{fontenc}
\usepackage[utf8]{inputenc}
\usepackage{lmodern}
\usepackage[pdfpagemode={UseOutlines},bookmarks=true,bookmarksopen=true,
bookmarksopenlevel=0,bookmarksnumbered=true,hypertexnames=false,
colorlinks,linkcolor={blue},citecolor={blue},urlcolor={blue},
pdfstartview={FitV},unicode,breaklinks=true]{hyperref}

\title{Antinet v0.1 - secure Meshnet with Smart Contract Micropayments for bandwidth and services}
\author{Anonymous Shamir}

\begin{document}

\maketitle
\tableofcontents

\begin{abstract}

\textit{Decentralized Internet imlemented on ideas of Meshnet and Open-Transactions to form best Free Market of Digital Services Providers.}

"Antinet.org" - called in short Antinet - is an p2p anonymous \& pseudonymous secure Meshnet network, 
that uses Smart Contract and Tokens based Micropayments
to allow automatic micro payments for network bandwidth 
and for any other digital services.
\newline

Antinet protocol - is description of crypto operations that will be used by this network.
\newline

Antinet program - is the program that implements this protocol.
\newline

Antinet aims to implement following goals:
\begin{itemize}
  \item Secure operation (done by the Cjdns meshnet and by own cryptography).
  \item Meshnet network running on own infrastructure as well as on regular Internet (done by the Cjdns).
  \item \textbf{Micro payments based on cash tokens} between users - full \textbf{financial decentralization}, and \textbf{fast processing} in order of \textbf{100 payments/s} \underline{per user} 
  (\textbf{easily millions of transactions per second} in entire city-sized network).
  \item Free-market exchanges for tokens/currencies issued by any users.
  \item Exchange bots that allow investors to arbitrate prices and exchange rates for the currencies (tokens) created by other users.
  \item Optional perfect anonymity with blinded signature, e.g. for anonymous but verifiable by everyone voting.
  \item Optional post-Quantum encryption for perfect future secrecy.
  \item Smart Contracts with own scripting for advanced financial and voting instruments.
\end{itemize}

With this properties, we hope Antinet should one day replace the \newline
current network: monopolized, censored, under-surveillance network \newline
with new network: always-encrypted, completely decentralized, where users pay each other (or exchange bandwidth with friends for mutual help) \newline
based on free-market of best network prices and digital services, as well as voting.

\end{abstract}

\section{Introduction}
Network of p2p computers in which:
\begin{itemize}
  \item Computers together build Internet connections network, e.g. in form of any \textbf{meshnet}, like the \textbf(cjdns) meshnet.
  \item Each node can Issue and redeem tokens that represent a value (e.g. connect to \textbf{Bitcoin})
  \item Each node can run an exchange to allow swapping tokens of one type to tokens of another type on free-market like order book.
  \item Each node can offer data-transfer services in exchange for payment in tokens.
  \item Each node can be commenced by it's operator to buy best access to the network.
  \item the operator can also request other digital services, e.g. computing powers, storage space.
  \item likewise, each node can act as service provider offering some digital services.
\end{itemize}

\section{Overview}

\subsection{Layers}
Nodes communicate together, form agreements (e.g. to buy data transfer).
\begin{enumerate}
  \item Networking layer - provides node to node connections enough to exchange needed agreement information. Communication is signed and encrypted. This can be implemented by cjdns.
  \item Tokens layer - users can issue each his own token / tokens. They can later reissue them - move them from one user to another, only at request of the current owner of coin. This is anonymous (e.g. RSA blinded signature tokens).
  \item Agreements layer - agreement sides are known by Nyms (Nym - a pseudonym) can talk to each others, sign contracts, and prove to others if they are cheated (in all important cases).
  \item Trust layer - Nym can publish or tell to selected users list of whom he trusts and who he does not trust. He can attach digital proves and evidence, e.g. that he was cheated by another Nym.
\end{enumerate}

\subsection{Use-case: buy any network access}

The User buys Internet access: 
he checks prices (and currencies) require by Internet providers,
he picks best one,
buys the currency (e.g. using Bitcoin and then exchanges), and pays that provider for each portion of data.

If situation changes: some other provider offers better deal, or the currency rate changes so that the final price is not optimal, 
the User can quickly switch to other provider.

This all creates a \textbf{decentralized, secure, optionally anonymous, perfectly free-market of network service providers}.

\section{Use-case: buy network access - detailed example}

This is detailed version of use-case summarized above.

User $U$ wants to buy network connection, to go to the regular (ICANN) Internet.

\begin{enumerate}
  \item He can directly connect to few Antinet nodes, they offer further connection e.g. "outside, to Hyperboria" at various prices in various currencies.

  \item User looks at currencies exchanges to find out the rates at which given currencies can be converted from what he has. User checks can he get the currencies needed and calculates the real price for him (how much e.g. Bitcoin he needs to spend in the end).

  \item User can use Bitcoins to get tokens of one currency, then exchange it on exchange-market to other currency, until he has the currency requested by the provider (ISP) he wants to buy from.

  \item He gets some amount of traffic for free usually, and later he pays e.g. for $1/100th$ of the amount needed, and then pays again and again each minute to minimize loses in case of fraud by ISP.

  \item Payment is done by giving the ISP his token, e.g. in blinded-signature-RSA-online-transfer operation.
\end{enumerate}

User $U$ gets list of nearby providers: 
\begin{itemize}
  \item cjdns providers that are near him called: $A$, $B$, $C$
  \item cjdns providers that are not reachable directly by him: $D$ .. $L$
  \item token issuers (in exchange for \textbf{Bitcoin}) near him: $X$, $Y$, and distant from him $Z$, and distant $W$
  \item issuers have pricing expressed in other tokens:
  $A_{price}=10\ W_{token}\ per\ 1\ MiB$ and 
  $B_{price}=15\ W_{token}\ per\ 1\ MiB$ and 
  $C_{price}=20\ W_{token}\ per\ 1\ MiB$
  \item to-Internet VPN gateway providers $S$ and $T$
\end{itemize}


\section{Crypto: blinded-signature-RSA-online-transfer}

\subsection{Conditions}

Alice $A$ wants to give coin to Dave $D$ know by his Pubkey. $A$ and $D$ are users of this.

This uses RSA blind signatures (and possibly also RSA encryption \& signing to protect the communication).

Pre-conditions: 
\begin{enumerate}
  \item All parties are communicating over signed and encrypted channels (e.g. the CJDNS) therefore additional encryption is not needed.
  \item Alice and Dave decided and want to execute this transfer.
  \item Trust: Alice and Dave trust the Bank that Bank has trustworthy-mint, 
if not then they can each e.g. run proper \hyperref[sec:crypto_mint-test]{mint-test}.
  \item \textbf{Alice $A$ has ownership of a coin}, that is expressed by her knowing (and keeping in secret) this coin's serial-number $m_{alice}$.
  \item \textbf{Alice $A$ is in contact with Dave $D$}, they can send each other a message in signed and encrypted public-key communication.
  \item \textbf{The Bank $B$} must be "online" (it will interact during the exchange).
  \item \textbf{This coin comes from Bank called $B$} that can receive signed/encrypted public key communication, he is known by public key $(e_{bank},n_{bank})$.
  \item \textbf{This coin was issued by Bank's Mint called $mint$} that is the signing key for tokens of value 100 (in some currency): $d_{mint}$. 
  Bank's Mint is known by public key $(e_{mint},n_{mint})$.
\end{enumerate}

Post-conditions:  
\begin{enumerate}
  \item Goal: Dave will have the ownership of the coin, instead of Alice
  \item ...and both Alice and Dave can prove that this transfer did happened correctly, and that they both agree on some Contract that is now therefore in affect.
  \item Alice will not learn anything new (except random and/or strongly encrypted numbers that she can not decrypt, and the fact that they were online at this time) about Dave nor Bank.
  \item also Dave will not learn anything (ditto) about Alice nor Bank.
  \item also Bank will not learn anything (ditto) about Alice nor Dave.
\end{enumerate}

\subsection{Execution}

\begin{enumerate}
  \item All operations here are executed in body $[modulo\ n_{mint}]$, all parties know the public key of Mint, so they know the module number.
  \item Dave wants to obtain his own, signed by Mint $mint$, token that will be ${m_{dave}}^{d_{mint}}$
  \item Dave creates his blinding factor $r_{dave}$ and it's reverse ${r_{dave}}^{-1}$
  \item Dave prepares his blinded token ${m'}_{dave}$ that he calculated as ${m_{dave}}*{r_{dave}}^{e_{mint}}$
  \item Dave gives the blinded token to Alice, as part of a signed contract that has logical business meaning of "If... " [TODO signature prove vs anonymity]
  \item [TODO] Alice gives to bank hash of her coin along with Dave's blinded message (or hash of it?)
  \item Bank signs he will take the coin in next e.g. 24 hours and reserves it's hash of serial number
  \item Alice sends unblinded hash of her coin to Bank
  \item Bank marks as fully spent, and blind-signs new coin of Dave, sends result to Alice and/or Dave
\end{enumerate}

\subsection{Attacks}

[TODO]



\section{crypto: mint-test for RSA mints}
\label{sec:crypto_mint-test}

Bank has a Mint, that is an RSA-blind-signature based mint - that is an RSA keypair used for RSA blind signatures.
This Mint consists of secret $d$ and $\varphi(n)$, and public $(e,n)$.
Alice wants to make sure that this is a proper trustworthy mint, that is:
\begin{enumerate}
  \item it can not silently "mark" the token, it can not sign one blinded-token in some special way that will later allow him to notice it is same token, when the token (unblinded) is presented back to him (or, any attempts to do this will be always detected by the person taking the coin, in a way were he can prove to others that the coin is wrongly signed)
  \item everyone can indeed check if the blind-signed message was signed by this mint
\end{enumerate}

Alice must NOT learn about the secret key nor any of its parts or hints allowing to reconstruct it - that is the number $d$ nor $\varphi(n)$.

It seems [Credit: CodesInChaos] [TODO: prove it] that there is exactly one possible blinded signature ($s'$) for each blinded-plain-text ($m'$) created from plain-text ($m$), 
if all following conditions are meet:
\begin{enumerate}
  \item $GCD(e,\varphi(n))==1$ - so the RSA key is propery generated
  \item $GCD(m, n)==1$
  \item $GCD(m',n)==1$    
\end{enumerate}

< CodesInChaos> If GCD(e,phi(n))==1 and GCD(m, n)==1 then there is exactly one possible signature for each plaintext.

Since the $m$ is controlled by the external user (Alice) of given RSA mint, then points 2 and 3 are easily solved.

Only issue is the condition 1 that $GCD(e,\varphi(n))==1$.


This happens always by definition when the key is any RSA key generated with normal RSA procedure. 
Then this will work, and it can sign given message $m$ in only one way giving
always the same number $m^{d}$ so it can not be used to mark coins.

And of course any properly created RSA key produces valid signatures, that can be checked as usual.

So we only need to prove that the Mint is normal RSA key that satisfies condition $GCD(e,\varphi(n))==1$ .

We can test for this in following way: it seems that [TODO: prove it] that any key that does not satisfy $GCD(e,\varphi(n))==1$ is an abnormal RSA "key", 
that always will malfunction in following way that it will fail to decrypt many of messages encrypted to it.

This means that if Alice could "interrogate" the Bank by sending it some number of messages, if Bank fails to decrypt any of them
then Bank is assumed to have an malicious RSA key and that mint (and in turn entire Bank) should be not trusted.

For a random test message $x$, there is probability $p$ that it will be not possible (without doing large amount of work - in order of magnitude like breaking the RSA key by 3rd party,
or maybe even at all? [TODO]) to decrypt that message by $Bank$.

\textbf{Security-warning:} Bank must avoid being tricked by this test, as bank decrypting the messages could be in fact signing some message (since signature/decryption are analogous operations in RSA).
One solution could be to agree that the test-message always have given length (integer value range, of the raw RSA numbers) 
other then the real token serial numbers (\hyperref[sec:formats_RSA-blind-signature-tokens]{see Formats}).





\section{Formats}

\subsection{RSA-blind-signature-tokens}

\label{sec:formats_RSA-blind-signature-tokens}

In RSA-blind-signature tokens, the messages for testing RSA Mint (\hyperref[sec:crypto_mint-test]{mint-test}) 
are raw integers for RSA in range e.g. $2^{512}\ ...\ 2^{768}$, \newline
and the RSA-blind-signature token serial numbers should be in range e.g. $2^{1024}\ ...\ 2^{2048}$ unblinded, \newline
and $2^{2049}\ ...\ 2^{4096}$ when blinded.

[TODO: verify this idea]



\end{document}
